\subsection{Pixelado diferencial}
El efecto de pixelado diferencial divide la imagen en cuadrados de 4x4 pixeles, y determina si se deben copiar los 16 pixeles originales o el promedio de estos, según si la diferencia absoluta entre el cuadrado de 4x4 y el promedio supera un umbral dado por parámetro. \\

\medskip

El filtro se aplica a todos los pixeles, en este caso se procesa de a 16 pixeles, porque en caso de tener que pixelar les corresponden los mismos valores a cada pixel. 
Como se utilizan imágenes de un ancho múltiplo de 8, no hay problemas a la hora de recorrer cada fila. Pero puede o no haber problemas en las últimas filas de la imagen. Si la cantidad de filas no es múltiplo de 4, que se puede chequear con ayuda de los datos de entrada y agregar saltos para resolver ese caso particular dependiendo de las filas que queden por recorrer. \\
\subsubsection{Pseudocódigo del ciclo:}
\begin{codesnippet}
\begin{verbatim}
Para i de 0 a height - 1;
    Para j de 0 a width - 1; 
        -- Paso 1 Promedio pixeles
        r=0, g=0, b=0;
        Para ii de i a i+3
            Para jj de j a j+3
                r = r + src_matrix[ii][jj].r
                g = g + src_matrix[ii][jj].g
                b = b + src_matrix[ii][jj].b
        b = SATURAR(b/16)
        g = SATURAR(g/16)
        r = SATURAR(r/16)
        -- Paso 2 Calculo de diferencia
        value = 0
        Para ii de i a i+3
            Para jj de j a j+3
                value += abs(r - src_matrix[ii][jj].r) + abs(g - src_matrix[ii][jj].g) 
                + abs(b - src_matrix[ii][jj].b);
        Paso 3 Aplicacion segun umbral
        Si value < limit:
            Para ii de i a i+3
                Para jj de j a j+3
                    dst_matrix[ii][jj].b = src_matrix[ii][jj].b
                    dst_matrix[ii][jj].g = src_matrix[ii][jj].g
                    dst_matrix[ii][jj].r = src_matrix[ii][jj].r
        Sino:
            Para ii de i a i+3
                Para jj de j a j+3
                    dst_matrix[ii][jj].b = b
                    dst_matrix[ii][jj].g = g
                    dst_matrix[ii][jj].r = r
        j = j+4;
    i = i+4;
\end{verbatim}
\end{codesnippet}

\subsubsection{Idea general del ciclo en ASM}

En cada ciclo se cargarán en 4 registros \textbf{XMM} 16 bytes de la imagen correspondientes a 4 filas consecutivas, es decir 4 pixeles en cada registro. \\
Luego se desempaquetan a word tanto su parte alta como baja de los registros \textbf{XMM}, por lo que se tendrán 8 registros \textbf{XMM} con los valores de 2 pixeles cada uno.\\

{\centering\textbf{Cálculo del promedio:}}

Para el promedio simplemente se suman todos los registros con la instrucción \textbf{PADDW}, se suma la parte alta con la parte baja del registro y se divide por 16 la word usando \textbf{PSRLW} y el inmediato con el valor 4, que serán la cantidad de bits que cada word se shifteará a derecha.
Finalmente se copia la parte baja del registro con los promedios a su parte alta.

{\centering\textbf{Cálculo de la diferencia:}}

Se utilizará un registro \textbf{XMM} como acumulador, realizando una sumatoria de los resultados de varias operaciones.\\
Estas operaciones consistirán en calcular la diferencia del promedio con cada uno de los 16 pixeles, para ello habrá una copia de este y los pixeles para conservar su valor original, y luego se procederá a restarle uno de los pixeles extendidos a 8 bytes, se toma el valor absoluto y se suman las componentes entre sí para finalmente sumar el resultado al registro acumulador usando \textbf{PADDW}. \\
Esto para cada uno de los 16 pixeles, en realidad 8 de estos a la parte baja y los otros 8 a la parte alta del registro del promedio.  \\
Finalmente se suman la parte alta con la parte baja del promedio, terminando con la acumulación. \\

{\centering\textbf{Aplicación según umbral:}}

Se empaqueta el registro promedio de word a byte sin signo y con saturación. \\
Usando \textbf{MOVD} y \textbf{PSLLDQ} se ubicará cada valor de value (registro acumulador) en los 2 quad words del registro, lo mismo se hará con el dato de entrada entero \textbf{limit}. \\
De esta forma haciendo pcmpgtq se obtendrá un registro repleto de 1's o 0's, dependiendo si value es mayor a \textbf{limit} o menor en ese orden. \\
Luego hago un \textbf{PAND} entre los promedios y la máscara obtenida. 
Continuamos haciendo \textbf{PANDN} con la máscara y los pixeles originales en caso de que la condición no se cumpla. \\
Se combinan los resultados haciendo un \textbf{POR}, obteniendo los valores pixelados o lo originales según el resultado de las comparación.\\
Por último se escribe a memoria de a 4 pixeles por fila realizando \textbf{MOVDQU}, se repite el ciclo o se termina.
